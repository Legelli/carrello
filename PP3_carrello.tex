%======================= PREAMBOLO DICHIARAZIONI INIZIALI===============%
%il mio computer è stronzo e non mi fa scaricare nessun pacchetto questi sono gli unici che riesco ad usare quindi poi aggiungete quello che vi serve 
\documentclass[10pt,oneside,a4paper]{article}

\usepackage[english]{babel}
\usepackage{amsmath}
\usepackage{graphicx}
\setcounter{section}{-1}

%========= PRIMA PAGINA ===========%
\title{\textsc{Misura dell'accelerazione di gravità attraverso lo studio del moto di un carrello su di un piano inclinato }}
\author{\small{G. Galbato Muscio} \and \small{L. Gravina} \and \small{L. Graziotto} \and \small{M. Rescigno}}
\date{}

\begin{document}
	\begin{figure}%
		\centering
		\includegraphics[scale=0.5,trim={2.8cm 8.9cm 0 9cm},clip]{logo.png}
	\end{figure}
	\maketitle
	\begin{center} 
		\fbox{{\fontsize{12pt}{8mm}\textsc{Gruppo B2.3}}} \\
		\vspace{1cm}
		\begin{tabular}{ccc}
			Esperienza di laboratorio && Consegna della relazione \\
			\emph{\small{20 aprile 2017}} && \emph{\small{2 maggio 2017}} \\
		\end{tabular} 
		
		\vspace{0.5cm}
		
	\end{center}
\hrule
\vspace{0.5cm}
\begin{abstract}
	Studiando dal punto di vista cinematico il moto di un carrello lungo un piano inclinato scabro, diamo diverse stime dell'accelerazione di gravità $g$ e del coeficiente di attrito dinamico $\mu_d$.
\end{abstract}
\newpage
\tableofcontents %Indice
\listoftables %Indice delle tabelle
\listoffigures %Indice dei grafici
\pagebreak
\section{Convenzioni e formule}
In questa relazione verranno usate le seguenti convenzioni:
\begin{enumerate}
	\item sarà usato il punto [ $.$ ] come separatore decimale;
	\item l'approssimazione decimale della cifra $5$ sarà fatta per eccesso;
	\item al fine di migliorare la qualità dell'elaborazione dei dati, ogni grafico/istogramma prodotto a mano su carta millimetrata sarà riportato insieme al suo equivalente prodotto attraverso un software di analisi dati\footnote{In questo contesto i dati sono stati elaborati con il software di analisi \emph{R}.};
	\item al fine di snellire la relazione e migliorarne la leggibilità, riporteremo nel corpo del documento solamente le tabelle riepilogative e dedicheremo un'appendice finale alle tabelle contenenti tutte le singole misure e i singoli risultati. %
\end{enumerate}
Inoltre, si farà riferimento alle seguenti formule:
\begin{enumerate}
	\item media 
	\begin{equation}\label{eq:media}
	\bar{x} = \frac{1}{N}\sum_{i=1}^Nx_i;
	\end{equation}
	\item varianza
	\begin{equation}\label{eq:varianza}
	\sigma^2 = \frac{1}{N}\sum_{i=1}^N(x_i-\bar{x})^2;
	\end{equation}
	\item deviazione standard
	\begin{equation}\label{eq:deviazione}
	\sigma = \sqrt{\sigma^2}.
	\end{equation}	
\end{enumerate}

%===============SCOPO E DESCRIZIONE DELL'ESPERIENZA==============%
\section{Scopo e descrizione dell'esperienza}
\label{sec:description}
Un carrello che si muove lungo un piano inclinato di un angolo $\theta$ rispetto all'orizzontale è soggetto a diverse forze. Proiettando tali forze lungo gli assi le tre componenti che intervengono sono:
\begin{itemize}
	\item  forza peso che accelera il carrello:
		\begin{center}  
			\begin{equation}\label{eq:forzapeso}
 			 F_p=mgsin\theta
			\end{equation} 
   		\end{center} 
   	\item la reazione vincolare del piano bilanciata dalla componente perpendicolare della forza peso 			
		\begin{center}
		    \begin{equation}\label{eq:reazionevincolare}
			N=mgcos\theta
			\end{equation}
		\end{center}
	\item la forza di attrito dinamico tra il carrello e il piano inclinato, con verso opposto al moto e che decelera il carrello:
		\begin{center}
			\begin{equation}\label{eq:forzaattrito}
			F_a=N\mu_d
			\end{equation}
		\end{center}
\end{itemize}
Dallo studio di queste forze sappiamo quindi che l'accelerazione totale del sistema è data da: 
\begin{center}
	\begin{equation}\label{eq:accelerazionetot}
	a_x=g(sin\theta \pm \mu_d cos\theta)
	\end{equation}
\end{center}
Dove il segno dipende dalla direzione dell'accelerazione (negativo nel caso discendente, altriementi positivo). Nel caso di piccoli angoli ($\theta \approx 0.18 rad$) si possono approssimare:
\begin{center}
	\begin{equation}\label{eq:approssimazione}
	sin\theta \approx \theta ; cos\theta \approx 1
	\end{equation}
\end{center}
Per cui la formula per l'accelerazione totale diventa:
\begin{center}
	\begin{equation}\label{eq:accelerazionepiccoliangoli)
	a_x=g(\theta \pm \mu_d).
	\end{equation}
\end{center}

In questa esperienza, dopo aver calibrato lo strumento per misure di posizione, daremo diverse stime di $\mu_d$ e di $g$ ottenute con più modalità:

\begin{itemize}
	\item misura di $g\mu_d$ con il piano orizzontale imprimendo una piccola velocità iniziale 
	\item misura di $g$ con un angolo fisso specifico 
	\item misura simultanea di $g$ e $\mu_d$ con angolo di inclinazione variabile 
	\item misura di $g$ ottenuta misurando tutto il moto sia nel tratto ascendente che nel tratto discendente
\end{itemize}
	
%================APPARATO SPERIMENTALE======================%		
\section{Apparato Sperimentale}
	
\subsection{Strumenti}
\label{subsec:strumenti}
\begin{itemize}
\item carrello;
\item Guida inclinata lunga circa due metri con scala graduata [divisione:1mm, incertezza:];
\item Sonar in grado di misurare la posizione del carrello in una dimensione a tempi diversi, interfacciato con il PC;
\item Programma Data Studio per calcolare derivando in modo numerico velocità e accelerazione;
\item Sqadra per misurare l'altezza del carrello [divisione:1mm, incertezza:];
\item livella.
\end{itemize}

%==============SEQUENZA OPERAZIONI SPERIMENTALI============%
\section{Sequenza Operazioni Sperimentali.}

\subsection{Verifica degli strumenti.}
\label{subsec:verifica}
Poichè il tavolo su cui poggiava la guida era inclinato di circa -1 grado, l'errore è stato corretto ottenendo così un piano perfetamente orizzonatale necesario per lo svolgimento delle prime misurazioni. Il sonar è stato invece calibrato seguendo le modalità indicate, come meglio approfondito nella sezione ~\ref{subsec:calibrazione delle misure di lunghezza dello stumento}.

%================ CALIBRAZIONE ==================%
\subsection{calibrazione delle misure di lunghezza dello strumento.}
Lo strumento utilizzato per registrare le misure di posizione (Sonar) deve essere calibrato. Per fare ciò è necessario posizionare il carrello sulla guida orizzontale e confrontare in un grafico le distanze misurate con il sonar e quelle di riferimento misurate con la scala millimetrata fissata alla guida. Dal grafico è possibile estrapolare la retta di calibrazione.\newline
\begin{center}
	\begin{equation}\label{eq:rettacalibrazione}
	x_riferimento=\alpha x_misurata+ \beta
	\end{equation}
\end{center}


%=============== MISURA DI G-Mu==========%
\subsection{Misura di $g\mu_d$.}
Mantenendo il piano sempre in posizione orizzontale sono state relaizzate misure ripetute di posizione imprimendo al carrello una piccola velocità iniziale. Poichè l'angolo di inclinazione è nullo ($\theta=0$), dopo aver ottenuto i valori dell'accelerazione in diversi istanti attraverso il programma Data Studio, è possibile ricavare diverse stime di $g\mu_d$. 
\begin{center}
	\begin{equation}\label{eq:accelerazione1}
	a_x=g\mu_d
	\end{equation}
\end{center}

%=================MISURA G ANGOLO SPECIFICO=============%
\subsection{misura dell'accelerazione di gravità con la misura dell'accelerazione ad un angolo specifico.}
Posizionando la guida inclinata ad un angolo $\theta$ abbstanza grande, in modo da ottenere una maggiore precisione, sono state relaizzate, sempre grazie al programma Data Studio, numerose stime (10 misurazioni da 10 secondi) dell'accelerazione mantenendo l'angolo $\theta$ fisso. Combinando i risultati ottenuti con le stime precedenti di $g\mu_d$ è possibile ottenere una stima delL'accelerazione di gravità $g$. 


%=============== ANGOLO VARIABILE========%
\subsection{Determinazione simultanea dell'accelerazione di gravità e del coeficiente di attrito variando l'angolo del piano inclinato.}
Variado l'angolo di inclinazione del piano, effettuiamo una misura di posizione da $10$ secondi per ogni angolo scelto. Facendo un grafico dell'accelerazione in funzione dell'angolo di inclinazione è possibile estrarre l'accelerazione di gravità dal coeficente angolare della retta che meglio approssima i punti. 



%============ SU E GIU' ========%
\subsection{Determinazione dell'accelerazione di gravità dalla differenza dell'accelerazione nel tratto discendente e quella nel tratto ascendente.}
Nell'ultima misurazione invece, tenendo l'angolo fisso a circa $0.05$ rad è stata misurata l'accelerazione sia nel tratto ascendente che discendente fino alla fine del moto del carrello. Dalla differenza delle due accelerazioni possiamo realizzare una stima di $g$
	\begin{equation}
	ax_x1= g(\theta + \mu_d) 
	\end{equation} (ascendente)\newline
	\begin{equation}
	a_x = g(\theta - \mu_d)
	\end{equation} (discendente)
Da cui facendo la differenza delle due accelerazioni otteniamo una stima di $g$.





%==== CONSIDERAZIONI FINALI =========%
\section{Considerazioni finali.}




\newpage
\section{Appendice: tabelle e grafici}


\end{document}
