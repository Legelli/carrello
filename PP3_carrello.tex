%======================= PREAMBOLO DICHIARAZIONI INIZIALI===============%
\documentclass[10pt,oneside,a4paper]{article}

\usepackage[latin1]{inputenc} 
\usepackage[italian]{babel}
\usepackage{siunitx} %Inserisce automaticamente i dati con le unit� di misura correttamente formattate del SI (utilizzo: \SI{0.82}{m^2}, in generale \SI{misura con il punto decimale}{unit� di misura})
\sisetup{output-decimal-marker = {.}, separate-uncertainty = true, input-uncertainty-signs = \pm, detect-weight=true, detect-family=true} %per usare SI con il punto decimale
\usepackage{listings} %Per citare codice informatico formattandolo correttamente
\usepackage{amsmath}
\usepackage{graphicx}
\usepackage{geometry}
\usepackage{epigraph}
\usepackage{booktabs}	%tabelle migliorate
\usepackage{tablefootnote}	%note a pi� di pagina in tabella
\usepackage{threeparttable} %tabella con note a pi� di tabella
\usepackage{caption}	%descrizione per figure
\captionsetup{tableposition=top,figureposition=bottom,font=small} %setup descrizione
\usepackage{float}
\usepackage{esvect} %vettori
\usepackage{longtable} %tabelle lunghe


\setcounter{section}{-1}

%========= PRIMA PAGINA ===========%
\title{\textsc{Misura dell'accelerazione di gravit� attraverso lo studio del moto di un carrello su di un piano inclinato }}
\author{\small{G. Galbato Muscio} \and \small{L. Gravina} \and \small{L. Graziotto} \and \small{M. Rescigno}}
\date{}

\begin{document}
	\begin{figure}
		\centering
		\includegraphics[scale=0.5, trim={2.8cm 8.9cm 0 9cm}, clip]{logo.png}
	\end{figure}
	\maketitle
	\begin{center} 
		\fbox{{\fontsize{12pt}{8mm}\textsc{Gruppo B2.3}}} \\
		\vspace{1cm}
		\begin{tabular}{ccc}
			Esperienza di laboratorio && Consegna della relazione \\
			\emph{\small{20 aprile 2017}} && \emph{\small{2 maggio 2017}} \\
		\end{tabular} 
		
		\vspace{0.5cm}
		
	\end{center}
\hrule
\vspace{0.5cm}
\begin{abstract}
	Studiando dal punto di vista cinematico il moto di un carrello lungo un piano inclinato scabro, diamo diverse stime dell'accelerazione di gravit� $g$ e del coeficiente di attrito dinamico $\mu_d$.
\end{abstract}
\newpage
\tableofcontents %Indice
\listoftables %Indice delle tabelle
\listoffigures %Indice dei grafici
\pagebreak
\section{Convenzioni e formule}
In questa relazione verranno usate le seguenti convenzioni:
\begin{enumerate}
	\item sar� usato il punto [ $.$ ] come separatore decimale;
	\item l'approssimazione decimale della cifra $5$ sar� fatta per eccesso;
	\item al fine di migliorare la qualit� dell'elaborazione dei dati, ogni grafico/istogramma prodotto a mano su carta millimetrata sar� riportato insieme al suo equivalente prodotto attraverso un software di analisi dati\footnote{In questo contesto i dati sono stati elaborati con il software di analisi \emph{R}.};
	\item al fine di snellire la relazione e migliorarne la leggibilit�, riporteremo nel corpo del documento solamente le tabelle riepilogative e dedicheremo un'appendice finale alle tabelle contenenti tutte le singole misure e i singoli risultati; data la mole di misure effettuate con il sonar spesso saranno riportate solo le sintesi dei dati.
\end{enumerate}
Inoltre, si far� riferimento alle seguenti formule:
\begin{enumerate}
	\item media 
	\begin{equation}\label{eq:media}
	\bar{x} = \frac{1}{N}\sum_{i=1}^Nx_i;
	\end{equation}
	\item varianza
	\begin{equation}\label{eq:varianza}
	\sigma^2 = \frac{1}{N}\sum_{i=1}^N(x_i-\bar{x})^2;
	\end{equation}
	\item deviazione standard
	\begin{equation}\label{eq:deviazione}
	\sigma = \sqrt{\sigma^2}.
	\end{equation}	
\end{enumerate}

%===============SCOPO E DESCRIZIONE DELL'ESPERIENZA==============%
\section{Scopo e descrizione dell'esperienza}
\label{sec:description}
Un carrello che si muove lungo un piano inclinato di un angolo $\theta$ rispetto all'orizzontale � soggetto a diverse forze. Proiettando tali forze lungo un asse parallelo e uno perpendicolare allo spostamento, le tre componenti che intervengono sono:
\begin{itemize}
	\item la forza peso che accelera il carrello
		\begin{equation}\label{eq:forzapeso}
 			 F_p=m g \sin\theta;
		\end{equation} 
   	\item la reazione vincolare del piano, che bilancia componente perpendicolare della forza peso 			
	    \begin{equation}\label{eq:reazionevincolare}
			N=mg \cos \theta;
		\end{equation}
	\item la forza di attrito dinamico tra il carrello e il piano inclinato, con verso opposto al moto e che decelera il carrello
		\begin{equation}\label{eq:forzaattrito}
			F_a=N\mu_d.
		\end{equation}
\end{itemize}
Dallo studio di queste forze sappiamo quindi che l'accelerazione totale del carrello � data da: 
\begin{equation}\label{eq:accelerazionetot}
	a_x=g(\sin\theta \pm \mu_d \cos\theta)
\end{equation}
dove il segno dipende dalla direzione dell'accelerazione (negativo nel caso discendente, altrimenti positivo). Nel caso di piccoli angoli ($\theta \le \SI{0.18}{rad}$) si possono approssimare:
\begin{equation}\label{eq:approssimazione}
	\sin\theta \approx \theta ;\qquad  \cos\theta \approx 1 \notag
\end{equation}
per cui la formula (\ref{eq:accelerazionetot}) per l'accelerazione totale diventa:
\begin{equation}\label{eq:accelerazionepiccoliangoli}
	a_x=g(\theta \pm \mu_d).
\end{equation}

In questa esperienza, dopo aver calibrato lo strumento per misure di posizione, daremo diverse stime di $\mu_d$ e di $g$ ottenute con pi� modalit�:

\begin{enumerate}
	\item misura del prodotto $g \cdot \mu_d$ con il piano in orizzontale, imprimendo una velocit� iniziale;
	\item misura di $g$ con un angolo fisso specifico;
	\item misura simultanea di $g$ e $\mu_d$ variando l'angolo di inclinazione;
	\item misura di $g$ ottenuta misurando tutto il moto sia nel tratto discendente che in quello ascendente.
\end{enumerate}
	
%================APPARATO SPERIMENTALE======================%		
\section{Apparato Sperimentale}
	
\subsection{Strumenti}
\label{subsec:strumenti}
\begin{itemize}
	\item Guida inclinata lunga circa due metri con scala graduata [divisione: \SI{1}{mm}, incertezza:];
	\item sonar in grado di misurare la distanza del carrello a tempi diversi [frequenza: \SI{20}{Hz}];
	\item software \emph{Data Studio} per raccogliere ed elaborare in modo preliminare le misure del sonar;
	\item squadra [divisione: \SI{1}{mm}, incertezza:];
	\item carrello;
	\item livella digitale [digit: \SI{1}{�}].
\end{itemize}

%==============SEQUENZA OPERAZIONI SPERIMENTALI============%
\section{Sequenza Operazioni Sperimentali.}

\subsection{Verifica degli strumenti.}
\label{subsec:verifica}
La guida risultava inclinata di circa \SI{-1}{�} quando i piedi erano completamente abbassati, la causa principale � attribuibile ad un'inclinazione della pavimentazione che, misurata, risultava anch'essa di circa \SI{-1}{�}. %, l'errore � stato corretto ottenendo cos� un piano orizzontale necessario per lo svolgimento delle prime misurazioni.
Oltretutto, la guida non era perfettamente piana bens� leggermente a catenaria, infossata nel centro (probabilmente a causa di un'assenza di supporti intermedi); non � stato possibile risolvere l'errore (si sarebbe dovuta cambiare la struttura della guida).
Il sonar � stato invece calibrato seguendo le modalit� indicate, come meglio approfondito nella sezione~\ref{subsec:Calibrazione}.

L'attrito tra il carrello e la guida era poco efficace, questo ha reso evidenti i difetti di costruzione della guida stessa sul moto del carrello.

Il software \emph{Data Studio}, necessario per interagire con il sonar, permette di scegliere arbitrariamente il numero di cifre significative con le quali memorizzare i dati, per cui l'ampiezza del digit non � ben definita: assumiamo come incertezza sulle misure di posizione esclusivamente quella di tipo \textbf{A}.

%================ CALIBRAZIONE ==================%
\subsection{Calibrazione delle misure di distanza}
\label{subsec:Calibrazione}
Lo strumento utilizzato per registrare le misure di posizione (sonar) deve essere calibrato: esso � tendenzialmente soggetto ad errori di offset e di scala (la velocit� del suono dipende dai fattori ambientali). Per quanto detto nella sezione \ref{sec:description} la posizione del carrello ha un'importanza solo relativa: interessandoci della sua accelerazione e non della sua posizione assoluta (equazione \ref{eq:accelerazionepiccoliangoli}), l'offset non disturba le nostre misure indirette e l'unico parametro di calibrazione significativo � quello di scala; per misurare tale coefficiente � necessario posizionare il carrello sulla guida orizzontale e confrontare in un grafico le distanze misurate con il sonar e quelle di riferimento misurate con la scala millimetrata fissata alla guida. Dal grafico � possibile estrapolare la retta di calibrazione, la quale � descritta da un'equazione del tipo:
\begin{equation}\label{eq:rettacalibrazione}
	x_{r}=\alpha x_{m} + \beta
\end{equation}
dove $x_r$ � la distanza di riferimento e $x_m$ quella misurata, $\alpha$ � il coefficiente di scala e $\beta$ l'offset. Le misure sono state prese lasciando il sonar attivo per circa 10 secondi ad ogni misura, per un totale quindi di 200 (\SI{20}{Hz} $\cdot$ \SI{10}{s}) misure per posizione; sul grafico \ref{fig:calibrazione} sono stati riportati i valori medi per ogni distanza con la relativa incertezza $( \delta = \sigma / \sqrt{N})$, le misure sono sintetizzate nella tabella \ref{tab:calibrazione}. Estrapolando i parametri direttori della retta, risultano:
\begin{equation}\label{eq:parametri_calibrazione}
	\alpha = \SI{ 1.0361 \pm 0.0029}{}, \qquad \beta = \SI{ -0.351 \pm 0.0029}{m}, %Mettere i valori effettivi
\end{equation}
dove le incertezze sono state prodotte insieme ai parametri dal software $R$ con il metodo dei minimi quadrati.

\begin{table}
\caption{Misure di calibrazione}
\label{tab:calibrazione}
\centering
\begin{tabular}{c|cc}
\hline
$x_r [\SI{}{m}]$ & $x_m [\SI{}{m}]$ & $\delta$ [\SI{}{m}] \\
$\pm$ 0.00015 && \\
\hline
0.50000 & 0.52084 & 0.00012 \\
0.60000 & 0.61273 & 0.00003 \\
0.70000 & 0.70871 & 0.00008 \\
0.80000 & 0.80446 & 0.00002 \\
0.90000 & 0.89897 & 0.00007 \\
1.00000 & 0.99997 & 0.00001 \\
1.10000 & 1.09359 & 0.00008 \\
1.20000 & 1.19130 & 0.00005 \\
1.30000 & 1.29049 & 0.00005 \\
1.40000 & 1.38738 & 0.00002 \\
\hline
\end{tabular}
\end{table}

%=============== MISURA DI G-Mu==========%
\subsection{Misura di $g\mu_d$.}
Mantenendo il piano sempre in posizione orizzontale sono state relaizzate misure ripetute di posizione imprimendo al carrello una piccola velocit� iniziale. Poich� l'angolo di inclinazione � nullo ($\theta=0$), dopo aver ottenuto i valori dell'accelerazione in diversi istanti attraverso il programma Data Studio, � possibile ricavare diverse stime di $g\mu_d$. 
\begin{center}
	\begin{equation}\label{eq:accelerazione1}
	a_x=g\mu_d
	\end{equation}
\end{center}

%=================MISURA G ANGOLO SPECIFICO=============%
\subsection{misura dell'accelerazione di gravit� con la misura dell'accelerazione ad un angolo specifico.}
Posizionando la guida inclinata ad un angolo $\theta$ abbstanza grande, in modo da ottenere una maggiore precisione, sono state relaizzate, sempre grazie al programma Data Studio, numerose stime (10 misurazioni da 10 secondi) dell'accelerazione mantenendo l'angolo $\theta$ fisso. Combinando i risultati ottenuti con le stime precedenti di $g\mu_d$ � possibile ottenere una stima delL'accelerazione di gravit� $g$. 


%=============== ANGOLO VARIABILE========%
\subsection{Determinazione simultanea dell'accelerazione di gravit� e del coeficiente di attrito variando l'angolo del piano inclinato.}
Variado l'angolo di inclinazione del piano, effettuiamo una misura di posizione da $10$ secondi per ogni angolo scelto. Facendo un grafico dell'accelerazione in funzione dell'angolo di inclinazione � possibile estrarre l'accelerazione di gravit� dal coeficente angolare della retta che meglio approssima i punti. 



%============ SU E GIU' ========%
\subsection{Determinazione dell'accelerazione di gravit� dalla differenza dell'accelerazione nel tratto discendente e quella nel tratto ascendente.}
Nell'ultima misurazione invece, tenendo l'angolo fisso a circa $0.05$ rad � stata misurata l'accelerazione sia nel tratto ascendente che discendente fino alla fine del moto del carrello. Dalla differenza delle due accelerazioni possiamo realizzare una stima di $g$
	\begin{equation}
	ax_x1= g(\theta + \mu_d) 
	\end{equation} (ascendente)\newline
	\begin{equation}
	a_x = g(\theta - \mu_d)
	\end{equation} (discendente)
Da cui facendo la differenza delle due accelerazioni otteniamo una stima di $g$.





%==== CONSIDERAZIONI FINALI =========%
\section{Considerazioni finali.}




\newpage
\section{Appendice: tabelle e grafici}



\end{document}
